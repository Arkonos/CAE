\documentclass[10pt,a4paper,fleqn]{article}
\usepackage[utf8]{inputenc}
\usepackage{amsmath}
\usepackage{amsfonts}
\usepackage{amssymb}
\usepackage{graphicx}
\usepackage[left=2cm,right=2cm,top=2cm,bottom=2cm]{geometry}
\usepackage[hidelinks]{hyperref}
\usepackage[naustrian]{babel}
\usepackage{csquotes}
\usepackage{pbox}
\MakeOuterQuote{"}
\begin{document}
\setlength{\mathindent}{2cm}% you can set it back to 0cm (the default).
\begin{itemize}
	\item \textbf{Nennen Sie die Abschnitte eines typischen Produktlebenszyklus.}
	\item \textbf{Welche Prozesse finden in den einzelnen Phasen eines Produktlebenszyklus statt?}
	\item \textbf{Warum sollten die Entwicklungszeiten möglichst kurz sein?}
	\item \textbf{Welchen Einfluss hat die Produktkomplexität auf die Produktentwicklung?}
	\item \textbf{Warum spielt die frühe Phase in der Produktentwicklung eine entscheidende Rolle?}
	\item \textbf{Nennen Sie Anforderungen an einen effizienten Produktentwicklungsprozess.}
	\item \textbf{Nennen Sie die Schritte in einem Entwicklungsprozess nach VDI 2221.}
	\item \textbf{Was versteht man unter virtueller Produktentwicklung?}
	\item \textbf{Wie werden die Entwicklungsstufen der rechnergestützten Produktentwicklung unterschieden?}
	\item \textbf{Benennen Sie folgende Begriffe: CAD, E-CAD, CAE, CASE, CAT, MKS, FEM, CFD, VR, CAM, TPD, PDM, HIL, SIL, PLM, PLC, STEP, IGES, DMU, VMU, CAS, PPC, RC, NC, MC}
	\item \textbf{Was sind und wofür werden digitale Produktmodelle / digitale Prozessmodelle verwendet?}
	\item \textbf{Was ist Frontloading? Wie / warum wird es eingesetzt?}
	\item \textbf{Was ist Simultaneous Engineering? Wie / warum wird es eingesetzt?}
	\item \textbf{Was versteht man unter einem "Development Process" und einem "Development Workflow"?}
	\item \textbf{Wodurch unterscheiden sich ein "Development Process" und ein "Development Workflow"?}
	\item \textbf{Was ist der Unterschied zwischen einem DUM und einem VMU?}
	\item \textbf{Was versteht man unter einem horizontalen und einem vertikalen Datentransfer in Bezug auf die computergestützte Entwicklung mechatronischer Produkte?}
	\item \textbf{Nennen Sie Formate für den Austausch von Geometriedaten in CAx-Workflows.}
	\item \textbf{Erklären Sie die folgenden Begriffe in Bezug auf den Datenaustausch in CAx-Workflows: Integrität, Datenkonsistenz, Datenrobustheit, Kompatibilität.}
	\item \textbf{Was versteht man unter einem Produktmodell im Sinne der computergestützten Entwicklung?}
	\item \textbf{Nennen Sie Beispiele für hardwarebasierte und virtuelle Produktmodelle.}
	\item \textbf{Erklären Sie die Begriffe "physikalische Modellbildung" und "phänomenologische Modellbildung".}
	\item \textbf{Vergleichen Sie die Vorteile von computergestützter Simulation und hardware-basierter Entwicklung.}
	\item \textbf{Welche Verfahren zur Modellbildung werden bei folgenden Computer-gestützten Simulationsverfahren angewandt: FEM, MKS (MBS), 3D-CFD}
	\item \textbf{Nennen Sie Anwendungen in der Entwicklung mechatronischer Produkte für folgende computergestützte Simulationsverfahren: FEM, MKS (MBS), CFD}
	\item \textbf{Nennen Sie mind. 5 Beispiele für CAx-Workflows.}
	\item \textbf{Beschreiben Sie folgende CAx-Workflows: CAD – DMU, CAD – FEM, CAD – MKS (MBS).}
	\item \textbf{Was versteht man unter den Begriffen "Tesselierung" und "Meshing".}
	\item \textbf{Nennen Sie die Funktionalitäten von PDM-Systemen (jeweils Produkt-bezogen und Prozessbezogen).}
	\item \textbf{Erklären Sie die fünf Hauptfunktionalitäten von PDM-Systemen näher.}
	\item \textbf{Was versteht man unter Produktdaten (im Sinne der virtuellen Produktentwicklung)?}
	\item \textbf{Erklären Sie eine möglich Abfolge von Schritten bei der Änderung eines mechatronischen Produktes.}
	\item \textbf{Wie ist ein PDM-System generell aufgebaut?}
	\item \textbf{Was versteht man unter dem Begriff "Mechatronik"?}
	\item \textbf{Welche Disziplinen sind in die Entwicklung mechatronischer Produkte involviert?}
	\item \textbf{Was sind die Herausforderungen bei der Entwicklung mechatronischer Produkte im Vergleich mit konventionellen mechanischen Produkten?}
	\item \textbf{Erklären Sie einen beispielhaften Entwicklungsprozess mechatronischer Produkte anhand des V-Modells.}
	\item \textbf{Welche Komponenten beinhalten mechatronische Systeme? Wie sind diese Komponenten die verschiedenen Domänen zuzuordnen?}
	\item \textbf{Was sind Aktuatoren?}
	\item \textbf{Welche Arten von Aktuatoren gibt es?}
	\item \textbf{Was sind Sensoren?}
	\item \textbf{Welche Arten von Sensoren gibt es?}
	\item \textbf{Nenne Sie Anforderungen an Sensoren in mechatronischen Produkten.}
	\item \textbf{Nennen Sie Verfahren zur Weg- bzw. Winkelmessung bei mechatronischen Produkten.}
	\item \textbf{Was versteht man unter Signalverarbeitung in mechatronischen Produkten?}
	\item \textbf{Was versteht man unter Prozessdatenverarbeitung in mechatronischen Produkten?}
	\item \textbf{Was versteht man unter Echtzeitdatenverarbeitung in mechatronischen Produkten?}
	\item \textbf{Was versteht man unter "Hardware-in-the-Loop" Entwicklung?}
	\item \textbf{Was ist ein HIL-Simulator?}
	\item \textbf{Warum wird HIL angewandt? Wo sind die Vor- bzw. Nachteile dieser Entwicklungsmethode?}
	\item \textbf{Was versteht man unter "Software-in-the-Loop" Entwicklung?}
	\item \textbf{Warum wird SIL angewandt? Wo sind die Vor- bzw. Nachteile dieser Entwicklungsmethode?}
	\item \textbf{Was sind "Upper-CASE-Tools” bzw. "Lower-CASE-Tools”?}
	\item \textbf{Nennen Sie Anwendungsbeispiele für E-CAD.}
	\item \textbf{Skizzieren Beispielhaft ein mechatronisches System und beschreiben Sie die Komponenten und Funktionen.}
	\item \textbf{Erklären Sie den Unterschied zwischen steuern und regeln.}
	\item \textbf{Was versteht man unter CAE?}
	\item \textbf{Skizzieren Sie eine beispielhafte Prozesskette CAD-CAE und benennen Sie die Workflows und Datenflüsse.}
	\item \textbf{Was versteht man unter FEM? Wozu wird diese Methode eingesetzt?}
	\item \textbf{Worauf ist bei der Aufbereitung von Geometriedaten für die FEM-Simulation besonders zu achten?}
	\item \textbf{Was versteht man unter "Diskretisierung" in der Strukturmechanik?}
\end{itemize}
\end{document}